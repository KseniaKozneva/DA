\section{Выводы}

В этой лабораторной работе я реализовала стабильную поразрядную сортировку для пар «ключ-значение» с ключами в формате автомобильных номеров и на практике разобрала полный цикл алгоритма: обработка ключа поразрядно, подсчёт частот символов на каждом разряде, построение префиксных сумм и распределение элементов по итоговым позициям.

Главный технический вывод: оценка \(O(m \cdot (n + k))\) достигается благодаря фиксированной длине ключа и малому алфавиту. В моём варианте ключ состоит из 8 символов, два из которых — пробелы на известных позициях, поэтому фактически обрабатывается только \(m = 6\) разрядов, а мощность алфавита равна \(k = 256\). Это позволяет алгоритму работать предсказуемо по времени даже на больших входных данных. При этом дополнительная память составляет \(O(n)\) за счёт сортировки индексов, а не самих объектов.

Также стало понятно, как обеспечивается стабильность сортировки: при распределении элементов по итоговым позициям нужно обходить массив справа налево, иначе относительный порядок элементов с одинаковыми ключами нарушается. Кроме того, я научилась корректно обрабатывать символы через static\_cast<unsigned char>, чтобы избежать проблем со знаковым типом char при индексации массива счётчиков.

\pagebreak